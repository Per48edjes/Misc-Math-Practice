\documentclass{article}
\usepackage{amsmath,amssymb,amsthm}

\newtheorem*{claim}{Claim}
\newtheorem*{lemma}{Lemma}
\DeclareMathOperator{\lcm}{lcm}

\title{$\aleph_{0}$ Weekly Problem}
\author{Ravi Dayabhai \and Conrad Warren}
\date{2024-08-31}

\begin{document}

\maketitle

\section*{Problem}

Prove that if you add up the reciprocals of a sequence of consecutive positive integers, the numerator of the sum (in lowest terms) will always be odd. For example, $\frac{1}{7} + \frac{1}{8} + \frac{1}{9} = \frac{191}{504}$.

\section*{Solution}

First, we prove the following lemma that will be helpful later.

\begin{lemma}
  In any sequence of consecutive positive integers $(a_{1}, a_{2}, \ldots, a_{n})$, there is a unique integer $a_{*}$ with the highest power of $2$ in its prime factorization.
\end{lemma}

\begin{proof}
  Suppose, for the sake of finding a contradiction, that there exists at least two positive integers in the consecutive positive integer sequence $(a_{1}, a_{2}, \ldots, a_{n})$ sharing a maximal power of $2$ in their prime factorizations. 

  Without loss of generality, let $a_j$ and $a_k$ (with $j < k$) have the same, highest power of $2$ in their prime factorizations (say, $2^{m}$) and all intervening $a_l$ (for all $l$ such that $j < l < k$) having a smaller power of $2$ in its prime factorization.
  Then, $a_{j} = 2^{m}d$ such that $d$ is odd (since it is the product of only odd integers) and $a_{k} = 2^{m}(d + 1)$.

  $d+1$ is even and factoring out a $2$ from $d+1$ gives $a_{k} = 2^{m + 1} \left(\frac{d + 1}{2}\right)$, where $\frac{d + 1}{2}$ is an integer.
  Herein lies a contradiction --- it is assumed that $a_{j}$ and $a_{k}$ have the same, highest power of $2$ among the prime factorizations of all $a_i$ in the sequence.

  Reaching a contradiction, we have proven the lemma.
\end{proof}
 
Next, we turn our attention to proving the original claim stated by the \textbf{Problem}.

\begin{claim}
  If you add up the reciprocals of a sequence of consecutive positive integers, the numerator of the sum (in lowest terms) will always be odd.
\end{claim}


\begin{proof}

  Consider any sequence of consecutive positive integers $(a_{1}, a_{2}, \ldots, a_{n})$.
  Let $N = \prod_{i=1}^{n} a_{i}$. Then, 

  \begin{align*}
    \frac{1}{a_1} + \frac{1}{a_2} + \ldots \frac{1}{a_n} = \frac{\frac{N}{a_1} + \frac{N}{a_2} + \ldots + \frac{N}{a_n}}{N}.
  \end{align*}

  Letting $\lcm(1, a_{1}, a_{2}, \ldots, a_{n}) = N^{\prime}$, the numerator, $S$, in the resultant fraction is:

  \begin{align*}
    S = \frac{N^{\prime}}{a_1} + \frac{N^{\prime}}{a_2} + \ldots + \frac{N^{\prime}}{a_n}.
  \end{align*}

  Showing that $S$ is odd is sufficient to prove the claim.
  What remains to be shown is that there exists exactly one odd summand (of $S$), $\frac{N^{\prime}}{a_{*}}$, implying $S$ is odd.

  By the Fundamental Theorem of Arithmetic, each $\frac{N}{a_i}$ has a unique prime factorization: say $\frac{N}{a_i} = 2^{m_{i,1}}3^{m_{i,2}}5^{m_{i,3}}\cdots$ for all $i \in \lbrace 1, 2, \ldots, n \rbrace$.  
  Consider $\frac{N}{a_{*}} = 2^{m_{*,2}}3^{m_{*,3}}5^{m_{*,5}}\cdots$, where $a_{*}$ is the member of the sequence with the highest power of $2$ in its prime factorization. Note that $a_{*}$ is unique by the lemma established above.
  This implies a unique $m_{*,2} = \min \lbrace m_{i, 2} \mid i \in \lbrace 1, 2, \ldots, n \rbrace \rbrace$.
  Because $2^{m_{*,2}}$ can be factored out of all $\frac{N}{a_{i}}$, it follows that $\frac{N^{\prime}}{a_{*}}$ is the only odd summand of $S$, giving odd $S$.

\end{proof}

\end{document}
