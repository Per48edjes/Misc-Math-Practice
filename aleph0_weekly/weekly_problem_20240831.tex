\documentclass{article}
\usepackage{amsmath,amssymb,amsthm}

\newtheorem*{lemma}{Lemma}
\DeclareMathOperator{\lcm}{lcm}

\title{$\aleph_{0}$ Weekly Problem}
\author{Ravi Dayabhai \and Conrad Warren}
\date{2024-08-31}

\begin{document}

\maketitle

\section*{Problem}

Prove that if you add up the reciprocals of a sequence of consecutive positive integers, the numerator of the sum (in lowest terms) will always be odd. For example, $\frac{1}{7} + \frac{1}{8} + \frac{1}{9} = \frac{191}{504}$.

\section*{Solution}

First we prove a lemma that will be helpful later:

\begin{lemma}
  In any sequence of consecutive positive integers $(a_{1}, a_{2}, \ldots, a_{n})$, there is a distinct integer $a_{*}$ with the highest power of $2$ in its prime factorization.
\end{lemma}

\begin{proof}
  % TODO: Need to prove the above lemma here.
\end{proof}
 
Now we can turn our attention to proving the original claim stated by the \textbf{Problem}:

\begin{proof}

  Consider any sequence of positive integers $(a_{1}, a_{2}, \ldots, a_{n})$.
  Let $N = \prod_{i=1}^{n} a_{i}$.
  Then the sum of the reciprocals of this sequence of positive integers can be expressed as:

  \begin{align*}
    \frac{1}{a_1} + \frac{1}{a_2} + \ldots \frac{1}{a_n} = \frac{\frac{N}{a_1} + \frac{N}{a_2} + \ldots + \frac{N}{a_n}}{N}.
  \end{align*}

  Since $\lcm(a_{1}, a_{2}, \ldots, a_{n}) \cdot \gcd (a_{1}, a_{2}, \ldots, a_{n}) = N$,
  the numerator in the resultant fraction expressed in lowest terms, $S$, is:

  \begin{align*}
    S = \frac{N^{\prime}}{a_1} + \frac{N^{\prime}}{a_2} + \ldots + \frac{N^{\prime}}{a_n},
  \end{align*}

  where $N^{\prime} = \frac{N}{\gcd(a_{1} , a_{2}, \ldots, a_{n})}$. 

  By the Fundamental Theorem of Arithmetic, $\frac{N}{a_i}$ has a unique prime factorization, hence $\frac{N}{a_i} = 2^{m_{i,1}}3^{m_{i,2}}5^{m_{i,3}}\cdots$, for all $i \in \lbrace 1, 2, \ldots, n \rbrace$.
  We can express $N$ similarly: $N = 2^{M_1}3^{M_2}5^{M_3}\cdots$.
  
  What remains to be shown is that there exists exactly \textit{one} \textbf{odd} summand $\frac{N^{\prime}}{a_i}$ for all $i \in \lbrace 1, 2, \ldots, n \rbrace$, thereby resulting in \textbf{odd} $S$. 

  Consider $\frac{N}{a_{*}} = 2^{m_{*,1}}3^{m_{*,2}}5^{m_{*,3}}\cdots$ such that $m_{*,1} = \min \lbrace m_{i, 1} | i \in \lbrace 1, 2, \ldots, n \rbrace \rbrace$.
  Note that $a_{*}$ is unique by the lemma established above.
  Because $2^{m_{*,1}}$ can be factored out of all terms $\frac{N}{a_{i}}$, it is a divisor of $\gcd(a_{1} , a_{2}, \ldots, a_{n}$).
  It follows that $\frac{N^{\prime}}{a_{*}}$ is the only odd summand of $S$.

\end{proof}

\end{document}
