\documentclass{article}
\usepackage{amsmath,amssymb,amsthm}

\newtheorem*{lemma}{Lemma}
\DeclareMathOperator{\lcm}{lcm}

\title{$\aleph_{0}$ Weekly Problem}
\author{Ravi Dayabhai \and Conrad Warren}
\date{2024-08-31}

\begin{document}

\maketitle

\section*{Problem}

Prove that if you add up the reciprocals of a sequence of consecutive positive integers, the numerator of the sum (in lowest terms) will always be odd. For example, $\frac{1}{7} + \frac{1}{8} + \frac{1}{9} = \frac{191}{504}$.

\section*{Solution}

First, we prove a lemma that will be helpful later:

\begin{lemma}
  In any sequence of consecutive positive integers $(a_{1}, a_{2}, \ldots, a_{n})$, there is a distinct integer $a_{*}$ with the highest power of $2$ in its prime factorization.
\end{lemma}

\begin{proof}
  Suppose, for the sake of finding a contradiction, that there exists at least two positive integers in the consecutive positive integer sequence $(a_{1}, a_{2}, \ldots, a_{n})$ sharing a maximal power of $2$ in their prime factorizations. 
  Without loss of generality, let $a_j$ and $a_k$ (and $j < k$, without loss of generality) both have the same highest power of $2$ in their prime factorizations, say $2^{m_{\dagger}}$, and no intervening $a_l$ (i.e., $j < l < k$) having a higher power of $2$ in its prime factorization.
  Then, $a_{j} = 2^{m_{\dagger}}d$ such that $d$ is odd (since it is the product of all odd primes dividing $a_{j}$) and $a_{k} = 2^{m}(d + 1)$.
  But $d+1$ is even and factoring out a $2$ from $d+1$ gives $a_{k} = 2^{m_{\dagger} + 1} \left(\frac{d + 1}{2}\right)$.
  This results in a contradiction since it is assumed that both $a_{j}$ and $a_{k}$ have the same, highest power of $2$ in the prime factorizations among all $a_i$.
  Having found a contradiction, we have proven the lemma.
\end{proof}
 
Next, we turn our attention to proving the original claim stated by the \textbf{Problem}:

\begin{proof}

  Consider any sequence of positive integers $(a_{1}, a_{2}, \ldots, a_{n})$.
  Let $N = \prod_{i=1}^{n} a_{i}$.
  Then the sum of the reciprocals of this sequence of positive integers can be expressed as:

  \begin{align*}
    \frac{1}{a_1} + \frac{1}{a_2} + \ldots \frac{1}{a_n} = \frac{\frac{N}{a_1} + \frac{N}{a_2} + \ldots + \frac{N}{a_n}}{N}.
  \end{align*}

  Letting $\lcm(a_{1}, a_{2}, \ldots, a_{n}) = N^{\prime}$, the numerator in the resultant fraction expressed in lowest terms, $S$, is:

  \begin{align*}
    S = \frac{N^{\prime}}{a_1} + \frac{N^{\prime}}{a_2} + \ldots + \frac{N^{\prime}}{a_n}.
  \end{align*}

  By the Fundamental Theorem of Arithmetic, $\frac{N}{a_i}$ has a unique prime factorization, hence $\frac{N}{a_i} = 2^{m_{i,1}}3^{m_{i,2}}5^{m_{i,3}}\cdots$, for all $i \in \lbrace 1, 2, \ldots, n \rbrace$.
  We can express $N$ similarly: $N = 2^{M_1}3^{M_2}5^{M_3}\cdots$.
  
  What remains to be shown is that there exists exactly one odd summand (of $S$) $\frac{N^{\prime}}{a_i}$ for all $i \in \lbrace 1, 2, \ldots, n \rbrace$, thereby resulting in odd $S$. 

  Consider $\frac{N}{a_{*}} = 2^{m_{*,1}}3^{m_{*,2}}5^{m_{*,3}}\cdots$ such that $m_{*,1} = \min \lbrace m_{i, 1} | i \in \lbrace 1, 2, \ldots, n \rbrace \rbrace$.
  Note that $a_{*}$ is unique by the lemma established above.
  Because $2^{m_{*,1}}$ can be factored out of all $\frac{N}{a_{i}}$, it follows that $\frac{N^{\prime}}{a_{*}}$ is the only odd summand of $S$, yielding odd $S$.

\end{proof}

\end{document}
